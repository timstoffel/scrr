\chapter{Fazit}
% Zusammenfassung aller Teile 
% Nachmaliges Herausstellen der Ergebnisse 
% Ausblick auf anschließende, zukünftige Arbeiten 

\section{Ausblick}

Die Betrachtung der Leistungsfähigkeit und Sicherheit ist nicht vollumfänglich. Die Messung der Leistungsfähigkeit lassen sich erweitern. Dazu wäre eine Betrachtung des Skalierungsverhaltens über 21 sinnvoll. Manche Kandidaten, wie Kata, werden von Intel mitentwickelt. Daher kann untersucht werden, ob sich die Performanz auf einer AMD CPU unterscheidet. Für die Ausführung von Kata FC muss Docker in Version 18.06 verwendet werden. Es kann getestet werden, ob sich die Leistung bei unterschiedlichen Docker Versionen unterscheidet. Damit kann ausgeschlossen werden, dass einer der Kandidaten schlechtere Ergebnisse erzielt, weil eine alte Docker Version verwendet wird. Zusätzlich können zu den Messungen weitere Kriterien wie die Leistungsbetrachtung von Redis oder Festplatten hinzugefügt werden. Neben den betrachteten Kandidaten kann die Untersuchung um weitere Kandidaten ausgedehnt werden.
Die Bewertungen zur Sicherheit könnten mithilfe einer systematischen Untersuchung von Sicherheitslücken ausgeweitet werden.
Der Kriterienkatalog kann um den Bereich Langlebigkeit erweitert werden. Darin wird untersucht, wie Zukunftsfähig die verschiedenen Projekte bezüglich ihrer Finanzierung und Entwicklergemeinschaft aufgestellt sind. Für eine solche Bewertung hat Projekt Chaoss der Linux Fondation schon eine Sammlung von Metriken vorgenommen \cite[][]{chaoss}.


\section{Zusammenfassung}

In der Arbeit werden verschiedene Alternativen für den Containerbetrieb vorgestellt, ein Kriterienkatalog konzipiert und anhand diesem eine Entscheidungsanalyse durchgeführt. Alle betrachteten Kandidaten haben eine höher Sicherheitsbewertung als Docker und sind somit geeignet, um vertrauliche Daten besser zu schützen. Die Kandidaten Kata und Nabla sind zum Teil sogar schneller als Docker. Eine höhere Isolierung ist daher nicht unbedingt mit Performanzeinbußen verbunden. Wenn Systeme mit Hardwarebeschleunigung zur Verfügung stehen, ist Kata die empfehlenswerte Alternative. Kata ist zu den meisten Docker Funktionen und allen Images kompatibel und lässt sich daher unkompliziert in bestehende Systeme integrieren. Zukünftig könnten Lösungen mit Unikernels wie Nabla die bessere Wahl sein. Dafür müssten aber mehr Programme auf den jeweiligen Unikernel portiert werden.