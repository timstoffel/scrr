%*******************************************************
% Abstract in German
%*******************************************************
\begin{otherlanguage}{ngerman}
	\pdfbookmark[1]{Zusammenfassung}{Zusammenfassung}
	\chapter*{Zusammenfassung}
Steigender Wettbewerb fordert von den Unternehmen mehr Flexibilität im Betrieb ihrer Softwareprodukte. Dabei setzt sich die Containervirtualisierung, aufgrund ihrer unkomplizierten Automatisierung zunehmend durch. Docker ist dabei die meist genutzte Lösung für den Containerbetrieb. Durch den Wechsel von virtuellen Maschinen auf Container entstehen neue Angriffsvektoren. Zur Erhöhung der Sicherheit, werden seit 2018 weitere Lösungen entwickelt, die einen höheren Schutz bieten sollen.

Das Ziel dieser Arbeit ist ein Vergleich der Kandidaten Kata Containers, gVisor und Nabla Container nach der Methode der Entscheidungsanalyse mit Docker. Dabei werden die Kandidaten bezüglich ihrer Leistungsfähigkeit, Sicherheit und Benutzbarkeit begutachtet. Für eine Bewertung der Performanz werden die Kandidaten in verschiedenen Disziplinen gemessen. Die Beurteilung im Bereich der Sicherheit erfolgt, indem geprüft wird, wie die Alternativen den Bedrohungen der OWASP Docker Top 10 entgegenwirken. Im Bereich der Benutzbarkeit wird die Kompatibilität zu Docker und Kubernetes betrachtet. Abschließend wird eine Empfehlung für Kata Containers ausgesprochen.

\end{otherlanguage}
